\section*{8 Computing the Singular Value Decomposition} %{{{1

\subsubsection*{Some general facts about the SVD:} %{{{2
\begin{definition}(Singular values:)
	The singular values of the matrix $A\in\mathbb C^{n\times m}$ is the 
	square root of the eigenvalues of the square symmetric matrix $A^TA$.
\end{definition}
\begin{theorem}
	The singular values $\{\sigma_i\}$ of the matrix $A\in\mathbb C^{n\times m}$ 
	are non-negative and real.
\end{theorem}
\begin{proof}
	Assume that $v_i$ are the eigenvectors of $A^TA$ with the eigenvalues $\lambda_i$,
	then 
	\[
		\norm{Av_i}_2^2 = v_i^\dagger A^\dagger A v = v_i^\dagger \lambda_i v_i = \lambda_i. 
	\]
	meaning that $\sigma_i = \sqrt{\lambda_i}$ are real and non-negative.
\end{proof}
\begin{theorem}(The Singular Value Decomposition:)
	Let $A$ be an $m\times n$ complex matrix of rank $r$. Then there exist an $m\times m$
	orthogonal matrix $U$, an $n\times n$ orthogonal matrix $V$ and an $n\times m$ diagonal 
	matrix $\Sigma$ such that 
	\[
		A = U \Sigma V^\dagger,
	\]
	where $\diag (\Sigma) = (\sigma_1,\sigma_2,\dots,\sigma_r, 0,0,\dots,0,)$ and 
	$\sigma_1\ge\sigma_2\dots\ge\sigma_r\ge0$.
\end{theorem}
\begin{definition}
	The columns of $U$ are called the left singular vectors of $A$ and the columns and the 
	columns of $V$ are called the right singular vectors of $A$.
\end{definition}
The singular value decomposition of $A$ is not unique unless $\text{rank}(A)=n$. In general 
different $U$ and $V$ can be used. It can be shown that the reduced singular value decomposition
is unique. 
The reduced singular decomposition is the decomposition 
$A = \tilde U\tilde \Sigma\tilde V^\dagger$, where $\tilde\Sigma$ is an $n\times r$
matrix such that $\diag{}(\tilde\Sigma)$ consists of all the singular values and has none zeros.

%}}}2

\subsubsection*{The connection to the Schur Decomposition:} %{{{2

There are important relationships between the singular value decomposition of a matrix
$A$, and the Schur decompositions of the symmetric matrices 
\begin{equation}
	A^TA,\, AA^T \text{ and } \smatrix{0 & A^T \\ A & 0}
\end{equation}
Assume that $U^TAV=\diag(\sigma_1,\dots,\sigma_n)$. Then
\begin{align}
	&V^TA^TAV = \diag(\sigma_1^2,\sigma_2^2,\dots,\sigma_n^2)\in\mathbb R^{n\times n}
	\label{eq861}\\
	&U^TAA^TU = \diag(\sigma_1^2,\sigma_2^2,\dots,\sigma_n^2,0,\dots,0)\in\mathbb R^{n\times m}
\end{align}
Moreover, if we decompose $U$
\begin{align}
	U= 
\begin{matrix}
	\\
	\begin{matrix}
	[ & U_1 & U_2 & ]\\
	  &	n & m-n &
	\end{matrix}
\end{matrix}
\end{align}
and define the orthogonal matrix $Q\in\mathbb R^{(n+m)\times(n+m)}$
\begin{align}
	Q=\frac1{\sqrt 2}\smatrix{V & V & 0 \\ U_1 & -U_2 & \sqrt2U_2}
\end{align}
then
\begin{align}
	Q^T\smatrix{0 & A^T \\ A & 0 } Q 
	= \diag(\sigma_1,\dots,\sigma_n,-\sigma_1,\dots,-\sigma_n,\underbrace{0,\dots,0}_{m-n})
\end{align}

%}}}2

\subsection*{8.6.1 Pertubation Theory and Properties}

\subsection*{8.6.2 The SVD Algori|thm} %{{{2

In this section a variant of the QR-algorithm is used to compute the SVD-decomposition of 
a matrix $A\in\mathbb R^{n\times m}$ with $m\ge n$.
Eq. \eqref{eq861} suggest that we
\begin{enumerate}[(i):]
	\item form $C=A^TA$,
	\item use the symmetric QR- algorithm to compute 
	$V_1^TCV_1 = \diag(\sigma_1^2,\sigma_2^2,\dots\sigma_i^2)$,
	\item apply $QR$ with column pivoting to $AV_1$ obtaining $U^T(AV_1)\Pi=R$.
\footnote{$\Pi$ is the pivoting matrix.}
\end{enumerate}
Note that since $U$ and $(AV_1)$ are orthogonal, then $R$ must be as well meaning that $R=I$.
(It is easy to show that for a triangular matrix to be orthogonal it must be diagonal).
Now we have that $(\Pi^TU^T A V_1 = \Sigma)$

The multiplication $A^TA$ is unwise from a numerical standpoint since small elements $A_{ij}<\sqrt{\ts u}$
will disappear due to round off error.
\footnote{
Read Ex. 5.3.2 in G\&L to find out why this approach is inaccurate.
Ex: for $A^T=\smatrix{1 & 0 & 10^{-3} \\ 1 & 10^{-3} & 0}$, $A^TA$ is singular in float arithmetics.
}


%}}}2

%}}}1

% vim:foldmethod=marker
